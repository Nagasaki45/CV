%!TEX TS-program = xelatex
% Add 'print' to the square brackets to remove colors from this template
\documentclass[]{friggeri-cv}

\begin{document}
\header{tom}{gurion}
       {Non-verbal communication in virtual reality}


%-------------------------------------------------------------------------------
%	SIDEBAR
%-------------------------------------------------------------------------------

% In the aside, each new line forces a line break
\begin{aside}
\section{contact}
~
\href{mailto:t.gurion@qmul.ac.uk}{t.gurion@qmul.ac.uk}
\href{http://www.tomgurion.me}{www.tomgurion.me}
+44 7397 965017
London, UK
~
\section{programming}
~
\textbf{data analysis:} python and the scipy stack
~
\textbf{creative tech:} max, pure data, arduino, processing, unity, blender
~
\textbf{server side:} elixir with phoenix or plug, python with django or flask
~
\textbf{web:} javascript, vue.js, jquery, bootstrap
~
\textbf{tools:} git, docker
\end{aside}


%-------------------------------------------------------------------------------
%	EDUCATION
%-------------------------------------------------------------------------------

\section{education}

\begin{entrylist}

    \entry
    {Since 2016}
    {PhD in Media and Arts Technology}
    {Queen Mary University of London}
    {
      \textbf{Thesis title:} Exploring listeners non-verbal cues in multiparty social interactions using virtual reality

      \textbf{Supervisor}: Prof. Patrick Healey and Dr. Julian Hough

      What do we do as listeners in conversation?
      How our behaviour affects the conversational flow and turn taking?
      What are the behavioural differences between addressees and side-participants?
      This thesis aims to tackle these questions.
      It uses virtual reality technologies to explore listeners non-verbal cues in multiparty social interactions.
      Findings from this research can inform new technologies to use socially relevant non-verbal cues in communication.
      They can be used to implement conversational agents with appropriate listening behaviours.
      They might also enable hybrid approaches that automatically manipulate some of the behaviours of listeners in VR.
      These can open new possibilities to enrich the social life of those lacking social skills, or reduce the need to constantly present an undivided attention in mediated conversations.

      \textbf{Activities}
      \begin{itemize}
        \item
          Organising the Virtual Reality special interest group.
        \item
          Running data analysis workshops.
      \end{itemize}
    }

\end{entrylist}
\begin{entrylist}

    \entry
    {2011--2015}
    {\href{http://www.tomgurion.me/pdfs/MA.pdf}{MA} in Music Technology}
    {Bar-Ilan University}
    {
      \textit{Exceptional distinction (97\%)}

      \textbf{Thesis title:} \href{http://www.tomgurion.me/pdfs/Gurion - An Audio-Only Augmented Reality System for Social Interaction.pdf}{An Audio-Only Augmented Reality System for Social Interaction}

      \textbf{Supervisors:} Prof. Eitan Avitsur and Dr. Nori Jacoby.

      This thesis examines how an interactive environment might facilitate social interaction by developing and evaluating a novel system for joint music consumption by a group of users in the same place and time.
      The system uses relative locations measured using a Bluetooth signal, and generates an immersive personalized augmented musical environment that depends on the location of its participants.
      Two experiments tested the system in the context of a silent disco party, using the system's relative position signals as well as video tracking to evaluate the experience of users with and without prior acquaintance.
      The results showed that for both groups, the system promoted openness and increased the social interaction between users.

      \textbf{Activities}
      \begin{itemize}
        \item
          Organising the weekly seminars of the Music Technology programme.
      \end{itemize}
    }

\end{entrylist}
\begin{entrylist}

    \entry
    {2010--2011}
    {Diploma studies in Music Technology}
    {Bar-Ilan University}

\end{entrylist}
\begin{entrylist}

    \entry
    {2006--2011}
    {\href{http://www.tomgurion.me/pdfs/BSc.pdf}{BSc} in Mathematics and Music}
    {Haifa University}
    {
      \textbf{Activities}
      \begin{itemize}
        \item
          Participating in the excellence programme of the department of mathematics.
        \item
          Taking part in building and maintaining the computer lab of the music department.
          Working as a recording and mixing engineer in projects by the music and the art departments.
      \end{itemize}
    }

\end{entrylist}


%-------------------------------------------------------------------------------
%	EXPERIENCE
%-------------------------------------------------------------------------------

\section{experience}

\begin{entrylist}

    \entry
    {2014--2016}
    {Python management server developer}
    {\href{http://xtremio.com/}{XtremIO}}
    {
      Working on the management server of an enterprise storage array.
      Developing automated maintenance procedures such as system upgrades, hardware replacements, restores, migrations, etc..
      Implementing high-level and interactive communication protocols between clients and server.
    }

\end{entrylist}
\begin{entrylist}

    \entry
    {2012--2014}
    {Music cognition research assistant}
    {Freelance}
    {
      Generating stimuli, executing music cognition experiments, and analysing experiments data.

      Working with Prof. Zohar Eitan from Tel-Aviv University and Dr. Roni Granot from the Hebrew University.
      \begin{itemize}
        \item
          Generated audio stimuli according to requested mathematical formulas (e.g. Shepard tones).
        \item
          Developed GUI application to run music cognition experiments with visual stimuli.
        \item
          Analysed recorded music using audio features extraction tools.
      \end{itemize}
    }

\end{entrylist}
\begin{entrylist}

    \entry
    {2014}
    {Python and Django tutor}
    {Freelance and with \href{http://www.10x.org.il/}{Udi Oron}}
    {
      Teaching programmers in hi-tech companies and independent developers.
      Developing teaching materials as well as teaching and assisting and class.
    }

\end{entrylist}
\begin{entrylist}

    \entry
    {2014}
    {Full-stack web developer}
    {\href{http://www.shachar-web.co.il/}{Shachar web}}
    {
      Developing ERP web application with python and django on the server side and html, javascript and jquery on the client side.
    }

\end{entrylist}
\begin{entrylist}

    \entry
    {2009--2013}
    {Math tutor}
    {Freelance and at \href{http://high-q.co.il/}{HighQ}}
    {
      Teaching mathematics from junior high to high school including preparation for matriculation exams.
      Teaching BSc students linear algebra A and B, calculus A and B, discrete mathematics, probability, statistics and more.
      Teaching classes at HighQ toward 4 units matriculation exams.
    }

\end{entrylist}


%-------------------------------------------------------------------------------
% PUBLICATIONS
%-------------------------------------------------------------------------------

\section{publications}

Gurion, T. and Jacoby, N. (2013) Audio-Only Augmented Reality System for Social Interaction. In: Stephanidis C. (eds) HCI International 2013 - Posters’ Extended Abstracts. HCI 2013. Communications in Computer and Information Science, vol 373. Springer, Berlin, Heidelberg.


%-------------------------------------------------------------------------------
% CONFERENCES
%-------------------------------------------------------------------------------

\section{conferences}

\begin{entrylist}

    \entry
    {2013}
    {International Symposium on Technology and Society}
    {Toronto}
    {\href{http://www.slideshare.net/Nagasaki45/audioonly-augmented-reality-system-for-social-interaction}{Talk}}

\end{entrylist}
\begin{entrylist}

    \entry
    {2013}
    {HCI International}
    {Las Vegas}
    {\href{http://www.tomgurion.me/pdfs/HCI2013 poster.pdf}{Poster} presentation}

\end{entrylist}


%-------------------------------------------------------------------------------
% AWARDS
%-------------------------------------------------------------------------------

\section{awards}

\href{http://www.tomgurion.me/pdfs/U8 mathematic contest - silver prize.pdf}{Silver prize} in The 2009 Global U8 Consortium Mathematics Contest.

\end{document}
