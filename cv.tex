%!TEX TS-program = xelatex
%!TEX encoding = UTF-8 Unicode
% Awesome CV LaTeX Template for CV/Resume
%
% This template has been downloaded from:
% https://github.com/posquit0/Awesome-CV
%
% Author:
% Claud D. Park <posquit0.bj@gmail.com>
% http://www.posquit0.com
%
% Template license:
% CC BY-SA 4.0 (https://creativecommons.org/licenses/by-sa/4.0/)
%

%-------------------------------------------------------------------------------
% CONFIGURATIONS
%-------------------------------------------------------------------------------
% A4 paper size by default, use 'letterpaper' for US letter
\documentclass[11pt, a4paper]{awesome-cv}

% Configure page margins with geometry
% \geometry{left=1.4cm, top=.8cm, right=1.4cm, bottom=.8cm}

% Color for highlights
% Awesome Colors: awesome-emerald, awesome-skyblue, awesome-red, awesome-pink, awesome-orange
%                 awesome-nephritis, awesome-concrete, awesome-darknight
\colorlet{awesome}{awesome-orange}
% Uncomment if you would like to specify your own color
% \definecolor{awesome}{HTML}{CA63A8}

% Colors for text
% Uncomment if you would like to specify your own color
% \definecolor{darktext}{HTML}{414141}
% \definecolor{text}{HTML}{333333}
% \definecolor{graytext}{HTML}{5D5D5D}
% \definecolor{lighttext}{HTML}{999999}

% Set false if you don't want to highlight section with awesome color
\setbool{acvSectionColorHighlight}{true}

% If you would like to change the social information separator from a pipe (|) to something else
\renewcommand{\acvHeaderSocialSep}{\quad\textbar\quad}

% Set colour for hyperlinks (\href)
\hypersetup{colorlinks,urlcolor=awesome-orange}

%-------------------------------------------------------------------------------
%	PERSONAL INFORMATION
%	Comment any of the lines below if they are not required
%-------------------------------------------------------------------------------
% Available options: circle|rectangle,edge/noedge,left/right
% \photo[circle,edge,right]{./profile.png}
\name{Tom}{Leverstone}
\position{Senior Software Engineer}
% \address{London}

\mobile{07397965017}
\email{tleverstone@gmail.com}
\homepage{leverstone.me}
\github{nagasaki45}
\linkedin{tleverstone}
% \gitlab{your-gitlab}
% \stackoverflow{SO-id}{SO-name}
% \twitter{@twit}
% \skype{skype-id}
% \reddit{reddit-id}
% \medium{medium-id}
% \googlescholar{googlescholar-id}{name-to-display}

% \quote{``A witty and insightful quote" - Someone}

%-------------------------------------------------------------------------------

\begin{document}

% Print the header with above personal information
% Give optional argument to change alignment(C: center, L: left, R: right)
\makecvheader[C]

% Print the footer with 3 arguments(<left>, <center>, <right>)
% Leave any of these blank if they are not needed
% \makecvfooter
%   {\today}
%   {Tom Leverstone~~~·~~~CV}
%   {\thepage}

%-------------------------------------------------------------------------------

\cvsection{Summary}

\begin{cvparagraph}
  I'm a Senior Software Engineer eager to make a significant impact and seeking my next challenge, ideally in a role where I can grow as a technical leader. I've consistently developed and implemented strategic solutions aligned with business objectives, from launching an AI-powered Q\&A product in weeks to improving system performance and reducing cloud costs. I thrive in environments where I can own projects from inception to completion, collaborate across teams, and delight users. I'm excited to find a role that supports work-life balance, offers opportunities to mentor others, and values expanding my technical breadth through hands-on contribution.
\end{cvparagraph}

%-------------------------------------------------------------------------------

\cvsection{Skills}

\begin{cvskills}

  \cvskill
    {Languages}
    {Python, JavaScript, TypeScript}

  \cvskill
    {Backend}
    {FastAPI, Django, Flask}

  \cvskill
    {Frontend}
    {React.js, Next.js}

  \cvskill
    {AI}
    {Retrieval-Augmented Generation, Evaluation of AI Applications, LLM-as-a-Judge}

  \cvskill
    {Databases}
    {MariaDB, Solr, Milvus}

  \cvskill
    {Cloud}
    {AWS (S3, EC2, RDS, Elastic Beanstalk), DigitalOcean (Droplets, AppPlatform), Heroku/Dokku}

  \cvskill
    {CI/CD}
    {GitLab CI, GitHub Actions, TravisCI}

  \cvskill
    {Other}
    {Git and GitHub/GitLab, Docker and Docker Compose, Linux}

\end{cvskills}

%-------------------------------------------------------------------------------

\cvsection{Experience}

\begin{cventries}

  \cventry
    {Senior Software Engineer}
    {Mintel}
    {London}
    {January 2024 - Present}
    {
      \begin{cvitems}
        \item {Collaborated within a 3-4 member engineering team to rapidly develop an AI-powered question-and-answer product in a six-week timeframe, achieving first-to-market status. This involved implementing a Retrieval-Augmented Generation (RAG) system built on an existing Django service, utilizing OpenAI GPT-4-Turbo, Milvus for vector search, Solr for access control, and S3 for content storage. Led the data indexing pipeline development to read content from S3, enrich it with metadata, chunk it, generate embeddings, and index them in Milvus.}
        \item {Led the design and development of a new backend service (using FastAPI, AsyncIO, and LangChain) to power the product, enabling rapid feature iteration and improved scalability. The design utilizes Server-Sent Events (SSE) to stream typed JSON objects to the client, including citations, images, and other data beyond the answer text. It is backed by MariaDB for conversation storage and retrieval.}
        \item {Optimized the service's performance, achieving a 6x throughput increase by replacing the LangChain Expression Language with a native AsyncIO solution and \href{https://leverstone.me/blog/performance-profiling-in-python-tools-techniques-and-an-unexpected-culprit}{addressing a performance bottleneck in the OpenTelemetry OpenAI instrumentation library}. This optimization allowed for a 7x increase in user base without additional infrastructure resources.}
        \item {Initiated and led an innovation workshop that brought together diverse perspectives from across the company. The top-voted project from the workshop progressed to the product roadmap with support from the Product Development and UX teams, demonstrating my ability to bridge technical possibilities with business opportunities.}
        \item {Investigated vector databases to recommend and lead a migration from Milvus to Solr, resulting in a >50\% reduction in cloud infrastructure costs while maintaining performance benchmarks.}
        \item {Optimized the indexing pipeline, reducing processing time from three hours to under ten minutes by processing only the daily difference instead of the entire dataset. This improvement enabled a shift from weekly to daily data updates, providing users with more up-to-date market intelligence.}
        \item {Led the development and implementation of a comprehensive quality evaluation suite for AI-powered features, including answer relevancy, a \href{https://leverstone.me/blog/automating-groundedness-evaluation-in-rag-applications}{custom groundedness metric}, and traditional information retrieval metrics (Mean Average Precision and Normalized Discounted Cumulative Gains). This suite incorporated LLM-as-a-Judge techniques, using OpenAI GPT-4o, to automate the evaluation process and provide data-driven insights for continuous improvement, informing prompt engineering, model updates, content retrieval adjustments (e.g., reranking), and more.}
        \item {Enhanced system observability by integrating OpenTelemetry, Jaeger, and Sentry.}
        \item {\href{https://leverstone.me/blog/mentoring-toolkit}{Mentored} two junior software engineers, focusing on their personal growth and career aspirations, while also advising on software design principles and best practices.}
      \end{cvitems}
    }

  \cventry
    {Software Engineer}
    {Mintel}
    {London}
    {August 2021 - December 2023}
    {
      \begin{cvitems}
        \item {Developed and maintained a full-stack web application using Django, React, and TypeScript.}
        \item {Actively participated in all aspects of the Agile Scrum process, including sprint planning, daily stand-ups, sprint reviews, and retrospectives.}
        \item {Led the development of a Single Page Application (SPA) using React and TypeScript for consuming long-form content, including the design and implementation of backend APIs using Django.}
        \item {Improved code quality and team efficiency by enhancing documentation, improving testing (e.g., advocating for more idiomatic use of pytest), and refining the alerting system (Sentry).}
        \item {Collaborated effectively with adjacent teams, contributing code to accelerate feature development and resolve bugs in shared dependencies. This involved attending cross-team meetings and contributing to shared code repositories.}
      \end{cvitems}
    }

  \cventry
    {Python Back-end Developer}
    {XtremIO}
    {Israel}
    {July 2014 - September 2016}
    {
      \begin{cvitems}
        \item {Developed and maintained the management server for an enterprise storage array, focusing on automated maintenance procedures such as system upgrades, hardware replacements, restores, and migrations.}
        \item {Designed and implemented reactive command-line UI components that reduced operational errors by guiding users through complex operations with a clear and intuitive UI. These included progress bars with stage information, pushed from the backend, and utilities for initiating prompts from the backend.}
      \end{cvitems}
    }

\end{cventries}

%-------------------------------------------------------------------------------

\cvsection{Education}

\begin{cventries}

  \cventry
    {PhD in Computer Science}
    {Queen Mary University of London}
    {London}
    {September 2016 - August 2021}
    {
      \begin{cvitems}
        \item {\textbf{Thesis:} Head Movement in Conversation. Supervised by Prof. Patrick Healey and Dr. Julian Hough.}
        \item {Designed and ran software skills workshops: Intro to Code with Python/Processing, Data Analysis with Python and Pandas, and Version Control with Git.}
        \item {Served as a teaching assistant for ``Interactive Digital Multimedia Techniques'', ``Parallel Computing'', and ``Arts Application Programming'' modules.}
        \item {Organised the Virtual Reality Special Interest Group.}
      \end{cvitems}
    }

  \cventry
    {MA in Music Technology}
    {Bar-Ilan University}
    {Israel}
    {2011 - 2015}
    {
      \begin{cvitems}
        \item {\textbf{Thesis:} An Audio-Only Augmented Reality System for Social Interaction. Supervised by Prof. Eitan Avitsur and Dr. Nori Jacoby.}
        \item {Achieved exceptional distinction.}
      \end{cvitems}
    }

  \cventry
    {BSc in Mathematics and Music}
    {Haifa University}
    {Israel}
    {2006 - 2011}
    {
      \begin{cvitems}
        \item {Participated in the excellence programme of the Department of Mathematics.}
      \end{cvitems}
    }

\end{cventries}

%-------------------------------------------------------------------------------

\end{document}